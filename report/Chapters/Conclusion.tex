\chapter{Conclusion and Future Work}
The initial concept for the project was to develop a tool that can automate the generation of network data to apply to an anomaly-based intrusion detection system. The project therefore first concentrated on research around the fields of intrusion detection systems and virtualisation, while not much pertinent material was found about the latter, at least in the cybersecurity field, there were a substantial collection of papers found which related to the use of machine learning techniques in intrusion detection systems, therefore a vast chunk of time was saved by studying existing research. From this research, a decision tree classifier was utilised in the project. However, the larger job of automating the generation of data had limited research around it and so had to be designed from scratch. The project has realized all of its objectives, principal of which was developing the capacity to generate datasets from automated virtualised network data. 

The research into automating virtualised networks reveals that I may use the networks in a cybersecurity setting to reliably generate network attack data. The user can write attack scripts to run on this virtualised network to generate network data, the network data will build a personalised dataset. The report also shows that this network data can then be applied to an anomaly intrusion detection system. This project provides a usable framework to allow the user to develop more complex and sophisticated tools from the datasets generated.

The program was packaged into a straightforward command based interface to be utilised by the user. The user does not require high-level technical knowledge to generate data or a model.

One particular future development avenue would integrate the project into a more complex system to generate novel attack data. The system would implement a novel technique to attack a network, the attack could then be executed on the automated system once they are run. A large amount of network data would be generated which could bolster network defences against more devious threats.

Automating the extraction of the system logs from the target machine would open up the possibilities of generating network data from a wider range of exploits such as the Canary Red’s Atomic Tools. These attacks would assume that access had already been made to the target machine and would run relevant attacks and extract system log data to detect these threats.

Using system logs would involve an alteration to the data collection and data processing process currently in place. The system might have to collect more network features and would have to parse system logs into the dataset despite there not being a one-to-one mapping between network data and systems logs.

Since this project is aimed at developing a system which any person could feasibly use, future research could work on developing the ability to alter the amount of machines on the network, and to alter the systems which run upon the network. 

Implementing ensemble learning techniques instead of a decision tree could be considered in the future. This will help reduce the risk of over-fitting from a single decision tree. This project’s scope, however, did not touch upon the different machine learning algorithms which could be used in place of the decision tree however the user could alter the data modelling section of the project if they wished to change the classifier.

The project has fulfilled all of its aims, chief of which was developing the ability to generate datasets from automated virtualised network data. These aims have allowed the development of a framework of which to build upon. There are multiple different avenues with which to move the project forward, the most compelling of them would be based on developing tools which can be used alongside the project to develop novel attack data, much like the Metasploit attacks used in this project. 
