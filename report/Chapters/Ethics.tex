\chapter{Professional and Ethical Issues}
During the development of any project there are certain ethical standards which must be upheld. Since this project is almost entirely software based, great care was taken to ensure that all aspects of the project fell strictly within the guidelines of the British Computer Society Code of Conduct \cite{bcs}. Following the code of conduct ensures that I created the project in a legal and socially appropriate manner.

Under the 'Duty to Relevant Authority' section of the code of conduct, rule 3.e. states that:
\begin{quote}
    You shall NOT misrepresent or withhold information on the performance of products, systems or services (unless lawfully bound by a duty of confidentiality not to disclose such information), or take advantage of the lack of relevant knowledge or inexperience of others \cite{bcs}.
\end{quote}

In accordance with section 3c of the code, no claims were made in this report or in any documentation which aimed to mislead the reader/user or overestimate the ability of the program. This allows a potential user to be sure of what the program does before downloading or deciding to develop upon it. 

Under the 'Public Interest' section of the code of conduct, rule 1.b states that:
\begin{quote}
    You shall have due regard for the legitimate rights of Third Parties \cite{bcs}.
\end{quote}

As the project involves the use of many third-party libraries and software. Where any third-party libraries and programs have been used, I have made this clear in this report and in development in accordance with the code. Any work within the project is my own except where explicitly stated otherwise. The project relies heavily upon operating systems running within the virtual machines. The use of all operating systems was licensed when appropriate and the testing images were not provided in line with such licensing.

Under the 'Professional Competence and Integrity' section of the code of conduct, rule 2.d. states that:

\begin{quote}
    You shall ensure that you have the knowledge and understanding of Legislation and that you comply with such Legislation, in carrying out your professional responsibilities  \cite{bcs}.  
\end{quote}

The most pertinent law around cybersecurity in the United Kingdom is the Computer Misuse Act of 1990 \cite{cmua}. The law aims to define under what circumstances someone may be charged with a cybercrime. The law states that 
\begin{quote}
     A person is considered guilty of an offence if they cause a computer to perform any function with intent to secure access to any program or data held in any computer, the access they intend to secure is unauthorised; and they know at the time when they cause the computer to perform the function that that is the case \cite{cmua}.
\end{quote}
 The project does not violate the computer misuse act due to the computer being a set of virtual machines running on the users computer, the access is therefore authorised. The intent of the project is very clearly defined to be furthering the development of intrusion detection systems and all attack suites provided are open source third party frameworks.
 
Any software developer must ensure that any project undertaken abides by the Data Protection Act 2018 \cite{dpa}. As the act only applies to personal data which is kept by the developer the act does not necessarily apply here. However to ensure that all user data is protected any data generated within the project is stored locally, at a location designated by the user. No unnecessary information such as the users name, address and date of birth is taken. 

While this project makes use of user scripted network exploits and hacking. This falls within the definition of ethical hacking, which unlike regular hacking is legal. Ethical hacking is defined as the hacking of a system, with the express knowledge of the systems owner, in the hopes of either testing defences or finding vulnerabilities within the system\cite{ethhak}. This project utilises hacking tools to exploit a pre-configured system, with pre-defined vulnerabilities, hoping to generate network data to build defences for that system. This is done with the knowledge of the systems owner and to better equip that system with defences. 






