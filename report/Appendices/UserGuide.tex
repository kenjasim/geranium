\chapter{User Guide}
\section{Prerequisites}
You must have packer and virtualbox installed to run anything in \texttt{generate} command. This can be downloaded from \url{https://packer.io/downloads.html}. If you are using Ubuntu linux then you can install virtualbox and Packer by running
\begin{verbatim}
     sudo ./install_linux.sh
\end{verbatim}

Finally to install python dependencies for the project run:
\begin{verbatim}
    sudo python3 setup.py install
\end{verbatim}

If you are using Linux then the system must be able to read network data without using sudo, this can be achieved by running these two commands
\begin{verbatim}
    setcap cap_net_raw=eip /usr/bin/pythonX.X 
    setcap cap_net_raw=eip /usr/bin/tcpdump
\end{verbatim}
X.X stands for the python version used.

\subsection{Virtual Machines}
Once packer is installed you must have 2 virtualbox ova files. These are pre-configured virtual machines which can be exported from VirtualBox following these instructions \url{https://docs.oracle.com/cd/E26217_01/E26796/html/qs-import-vm.html}. You must either create an attack machine (Kali was used in testing) and a target machine (a Windows 7 machine was used in testing) or download the testing images. Creating a virtual machine is detailed further in the virtual box documentation. Full instructions can be found here \url{https://docs.oracle.com/cd/E26217_01/E26796/html/qs-create-vm.html}

\subsubsection{Testing Machines}

The testing virtual machines can be found at \url{https://www.icloud.com/iclouddrive/0mBc558CSjw7Cc2HFCMCjnbQw#virtual-machines}. If the gateway IP to your network is not 192.168.0.1 then you may need to change the IP addresses of the virtual machines.

\textbf{Attack} - For Kali Linux the static IP can be changed from the command line. First the file  \texttt{/etc/network/interfaces} needs to be altered. Inside that file these lines were added
\begin{verbatim}
    auto eth0
    iface eth0 inet static
    address 192.168.0.14/24
    gateway 192.168.0.1
\end{verbatim}
This alters the IP address of the \texttt{eth0} interface to 192.168.0.14, the IP address which will identify the attack machine. Once this alteration has been made the networking service must be restarted.
\begin{verbatim}
    sudo systemctl restart networking.service
\end{verbatim}

When the service has restarted, the IP should have changed as in Figure \ref{fig:IPfa}. This IP will stay as the IP address of the machine when the machine will be restarted/exported.

\textbf{Target} - For Windows 7 the static IP can be changed from the Control Panel. First the relevant section of the control panel should be accessed by Start Menu > Control Panel > Network and Sharing Center > Change adapter settings. From here the Local Area Network interface  Internet Protocol Version 4 (TCP/IPv4) properties are accessed, the IP address is then changed to include the static IP address. The static IP for the target machine will be 192.168.0.15. Once saved the IP address of the target machine will have been altered as seen in Figure \ref{fig:ipft}.

Next alter the relevant sections of the config file accordingly:
\begin{itemize}
    \item \texttt{attack\_machine\_path}: Location of attack machine
    \item \texttt{target\_machine\_path}: Location of target machine
    \item \texttt{attack\_username}: Attack machine ssh username
    \item \texttt{attack\_password}: Attack machine ssh password
    \item \texttt{attack\_IP}: Attack machine IP
    \item \texttt{filter\_IP}: IP of target machine to filter from
    \item  \texttt{interface}: The interface you are bridging
\end{itemize}

Other than the file-path the configurations in the config file for the attack and target machine can stay the same if there is no change to the IP address. The username and password should remain unchanged.

\section{Usage}

\subsection{Generate}
To generate data you must first create an attack to generate data from. For example using the metasploit framework a synflood attack may look like

\begin{lstlisting}
    service postgresql start

    msfdb init

    timeout 10m msfconsole -q -x "use auxiliary/dos/tcp/synflood;set RHOST <IP>; exploit;"
\end{lstlisting}
Next alter the config file:
\begin{itemize}
    \item \texttt{executable\_path}: Packer executable path
    \item \texttt{time}: Time in seconds to run the generation
    \item \texttt{attack\_machine\_path}: Location of attack machine
    \item \texttt{target\_machine\_path}: Location of target machine
    \item \texttt{attack\_username}: Attack machine ssh username
    \item \texttt{attack\_password}: Attack machine ssh password
    \item \texttt{attack\_IP}: Attack machine IP
    \item \texttt{filter\_IP}: IP of target machine to filter from
    \item  \texttt{interface}: The interface you are bridging
    \item \texttt{dataset\_path}: Location of the dataset, found under \texttt{data-processing}
\end{itemize}

Once an attack has been made and the config file altered correctly then you can use

\begin{verbatim}
    ./geranium.py generate synflood <path/to/synflood\_attack.sh>
\end{verbatim}

This will run for the allotted time as defined in the config file and will generate a CSV with features as defined in \texttt{data\_processing/data\_parser.py}.

\subsubsection{Generating Normal Network Data}

To generate normal data run:

\begin{verbatim}
    sudo ./geranium.py generate normal
\end{verbatim}
This used the web-traffic-generator from: \url{https://github.com/ecapuano/web-traffic-generator}. You only need to specify the time in the config file for this.
\subsubsection{Clearing the Virtual Machines}
On exit all the relevant folders for the virtual machines should have been removed, but if not you can run:

\begin{verbatim}
    sudo ./geranium.py clearvms
\end{verbatim}

\subsection{Model}

Once the generate command has been run you can use the data generated to create a decision tree model. An example decision tree is provided. This was generated using sklearn.

To generate a decision tree from the data, first alter the config file:
\begin{itemize}
    \item \texttt{model\_path}: Path to store model
    \item \texttt{classes}: Classes found in dataset (List)
\end{itemize}

From here a decision tree model can be generated using:

\begin{verbatim}
    ./geranium.py model <path/to/dataset>
\end{verbatim}

\subsubsection{IDS}

For a rudimentary intrusion detection system you can specify the model in the config file:
\begin{itemize}
    \item \texttt{model}: Place to locate the model
\end{itemize}
Then the intrusion detection system can be run with the command

\begin{verbatim}
    sudo ./geranium.py ids
\end{verbatim}
