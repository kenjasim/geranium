\chapter*{abstract}
\addcontentsline{toc}{chapter}{\numberline{}Abstract} 
An Intrusion Detection System (IDS) conventionally uses signature-based detection. This approach works by identifying specific patterns associated with an attack, such as byte sequences in network traffic to a database of known attacks. This approach has two major limitations. It requires extensive, expert lead configuration and cannot detect zero-day (novel) attacks.

A new approach to IDS involves anomaly-based detection, leveraging recent advancements on Data Mining and Machine Learning technologies to profile normal behaviour and then detect anomalies/indicators of compromise. Typically, this is implemented using algorithms such as K-Nearest Neighbour or Fuzzy logic). This technology has been successfully implemented at production level within financial fraud detection, but is still emerging within cyber-security. This approach requires a large amount of data, however, which is lacking in the cybersecurity community.

The goal of the project is to implement an automated visualised network that can generate a large amount of data from a wide range of configurable threats. This data will then train a decision tree model to create an Intrusion Detection Model. The technology will mainly rely on decision trees for their 'explainability'. Demonstrated success in this under-researched area would be an important step in cybersecurity systems.